\documentclass[russian, utf8, nocolumnxxxi, nocolumnxxxii]{eskdtext}
\usepackage[utf8]{inputenc}
\usepackage[T1,T2A]{fontenc}
\usepackage[english, russian]{babel}
\usepackage[argument]{graphicx}
\usepackage{graphicx}
\graphicspath{{images/}}
\usepackage{wrapfig}
\usepackage{tikz}
\usepackage{siunitx}
\usepackage[american,cuteinductors,smartlabels]{circuitikz}

\title{KR}
\author{danilaturygin}
\date{December 2018}
\begin{document}
1. Даны функции $f(x) = $\sqrt{3}$sin(x) + cos(x)$; \\
$g(x) = cos(2x + \pi/3) - 1$ \\
а)Решить уравнение $f(x) = g(x)$. \\
б)Исследовать функцию  $h(x) = f(x) - g(x)$ на промежутке $[0;$\frac{5}{6}$\pi]$\\
Строим график функции:\\
$function y = h(x)$ \\
$y = sqrt(3) * sin(x) + cos(x) - cos(2*x + pi/3) + 1$ \\
$endfunction$ \\
$plot(0:0.01:2*pi,h)$ \\
\\
$$Рисунок 1$$ \\
\\
Находим корни уравнения: \\
$deff$($'y$=h(x)'$,$'y=sqrt(3)*sin(x)+cos(x)-cos(2*x+$pi$/3)+1'$) \\
x0=[3,4.2,4.3,5.5]; \\
$[x,v]$=fsolve(x0,h) \\
 $v$  = \\
-2.220D-16 \quad 0. \quad 0. \quad 7.772D-16 \\
 $x$  = \\
   2.6179939 \quad 4.1887902 \quad 4.1887902 \quad 5.759586 \\
3*x/$pi$ \\
 ans  = \\
   2.5 \quad 4. \quad 4. \quad 5.5\\
Корни уравнения:\\
$$x_1 = \frac{5}{6}\pi + 2n\pi, n \in Z$$
$$x_2 = \frac{8}{6}\pi + 2n\pi, n \in Z$$
$$x_3 = \frac{11}{6}\pi + 2n\pi, n \in Z$$
\\
Исследование функции $h(x) = f(x) - g(x)$ на промежутке [0; $\frac{5}{6}$ \pi]\\
$(Reduse) $Упростим функцию$ $f(x)$ = $\sqrt{3}$sin(x) + cos(x) - cos(2x + $\pi$/3) + 1$, $получив$ $f(x)$ = 2sin(x+\pi/6)*(1+sin(x+\pi$/6))$\\
$Найдём значение функции $f(x)$ в точках $x=0$ и $x$=$\frac{5}{6}$$\pi$ с помощью оператора подстановки sub: sub(exp,f)=g, где g-результат, полученный при подстановке списка алгебраических выражений exp в функцию f.$\\
$h(x):=2sin(x+pi/6)*(1+sin(x+pi/6))$;\\
$sub(x=0,h1)$;\\
$sub(x=5pi/6,h2)$;\\
$Найдём первую и вторую производную с помощью оператора df: df(f,x,n) = dif, где dif – аналитическая форма производной n-го порядка для функции f по переменной x.$\\
$dh1:=df(h,x,h1)$;\\
$dh2:=df(h,x,h2)$;\\
$Построим графики первой и второй производной$:\\
$plot(dh1,x=(0..5pi/6))$;\\
$plot(dh2,x=(0..5pi/6))$;\\
(Scilab) Основываясь на виде графиков первой и второй производной найдём нули первой и второй производной, используя оператор $num$_$solve$: $num$_$solve(f,x = const) = f0$, где $f0$ – ноль функции $f$, найденный численно при начальной точке работы алгоритма $x = const.$\\
$Проведя необходимые вычисления, о функции, в исследуемом промежутке, можно сказать следующее:\\
1) Функция возрастает на промежутке (0,$\frac{\pi}{3}$);\\
2) Функция убывает на промежутке ($\frac{\pi}{3}$,$\frac{5}{6}$ $\pi$);\\
3) Имеет максимум в точке x = $\frac{\pi}{3}$;\\
4) Имеет минимум в точке $x = $\frac{5}{6}$ \pi $;\\
5) Точки перегиба функции x=0.111 и x=1.193;\\
6) Выпукла вверх на промежутках (0,0.111) и (1.193,$\frac{5}{6}$ $\pi$);\\
7) Выпукла вниз на промежутке (0.111,1.193);\\
\end{document}
Нахождение коэффициентов кубического сплайна:\\
x = [0,1.25,2.0,2.625,4.25]; \\
y = [3.0,2.925,3.75,3.42,4.444]; \\
d = splin(x,y); \\
n = length(x)-1; \\
cfs = zeros(4,n); \\
for i=1:4 \\
    a = x(i); \\
    b = x(i+1); \\
    $cfs(:,i)$ = $[1,a,a^2,a^3; 1,b,b^2,b^3; 0,1,2*a,3*a^2; 0,1,2*b,3*b^2]$ \ $[y(i);y(i+1);d(i);d(i+1)]$\\
end\\
cfs =\\
    3.\quad 3.\quad- 10.113334 \quad -10.113334  \\
  - 2.9084975 \quad- 2.9084975 \quad16.761504 \quad 16.761504  \\
    3.3405467 \quad 3.3405467 \quad- 6.4944538 \quad -6.4944538  \\
  - 0.849399 \quad -0.849399 \quad 0.7897678 \quad 0.7897678  \\
Первые два уравнения $1,a,a^2,a^3$; $1,b,b^2,b^3$ связывают значения многочлена с y-значениями; остальные два $0,1,2*a,3*a^2$; $0,1,2*b,3*b^2$ делают то же самое для своей производной. (Формулы являются просто производными от степени х). \\
Построим график функции (рисунок 2) на основе полученных данных: \\
$for i$ = 1:4\\
    $t = linspace(x(i),x(i+1))$; \\
    $plot(t,cfs(:,i)'*[ones(t); t; t.$^$2; t.$^$3])$ \\
$end$ \\
\\
$Рисунок 2$\\
\\
$Найдем значение функции в точке $x$=1.8:$\\
$X$=[1.8];\\
$Y=interp(X,x,y,d)$\\
$plot2d(X,Y,[-4])$;\\
$t$=0:0.01:4.445;\\
$ptd=interp(t,x,y,d)$;\\
$plot2d(t,ptd)$\\
\\
$Графическое изображение результатов интерполяции исходных данных с использованием функций:$\\
$splin(x,y,"natural"),$ \\
$splin(x,y,"clamped"),$ \\
$splin(x,y,"not_a_knot"),$ \\
$splin(x,y,"fast"),$ \\
$splin(x,y,"monotone")$ \\
$interp(xx, x, y, d)$ \\
Для того чтобы представить графическое изображение результатов интерполяции исходных данных, $с$ помощью вышеперечисленных функций, задаём известные точки графика функции ($x$ и $y$) и промежуток построения графика ($t$). Далее задаем параметр $d$, соответствующий одной из функций и значения интерполяционного полинома ($ptd$) в точке ($t$):$ \\
$x = [0,1.25,2.0,2.625,4.25];$\\
$y = [3.0,2.925,3.75,3.42,4.444];$\\
$t = 0:0.01:4.445;$\\
$d = splin(x,y,"parametr")$\\
$ptd = interp(t,x,y,d)$\\
$plot2d(t,ptd)$\\
$Таким образои были построены графические изображения результатов интерполяции
исходных данных с использованием необходимых функций:$\\
\\
\\
\newpage
Пусть матрица $A$ размерности $mxn$ - количество сырья имеющегося в распоряжении завода железобетонных изделий, где $m$ - виды сырья (песок, щебень, цемент) и продукция $n$ видов. Вектора $b$ и $c$ - количество ресурсов на заводе и прибыль $с$ единицы изделия соответственно. $ci$ - вектор размерности $n$ содержащий нижнюю границу переменных ($ci_j > x_j$), $cs$ - вектор длиной $n$, содержащий верхнюю границу переменных ($cs_j > x_j$).$\\
$c = [35;45;36;28];$\\
$A = [3 7 6 7; 4 5 5 1; 5 4 9 8];$\\
$b = [16;12;35];$\\
$ci = [0;0;0;0];$\\
$cs = [];$\\
$[x,kl,f]=linpro(-c,A,b,ci,cs)$\\
$f  =$\\
$- 126.56$ \\
$x  =$\\
2.72\\
0.\\
0.\\
1.12\\
$В результате, для получения наибольшей прибыли f = 126.56, необходимо произвести 2.72 ед. продукции первого и 1.12 ед. продукции второго вида.$\\